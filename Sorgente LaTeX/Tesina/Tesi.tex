\documentclass{tstextbook}

\begin{document}

\tsbook{\Huge }
       {Chiara Gobbi \\ Daniele Benfatto \\ Davide De Zuane \\ Began Bajrami}
       {Antonio Muscerino}
       {2021}
       {xxxxx}{xxx--xx--xxxx--xx--x}{1.0}
       {Università Politecnica delle Marche}
       {Ancona}

%---------------------------------------------------------------------------
% Chapters
%---------------------------------------------------------------------------

%---------------------------------------------------------------------------

\begin{center}
	\textbf{\Large Abstract}
\end{center}
Questo progetto nasce come prova svolta per il corso di \textit{Ingegneria del Software}di \textit{Ingegneria Informatica e dell'Automazione} presso l'\textbf{Università Politecnica delle Marche}, in particolare come prova di Ingegneria del Software.\\
Il progetto nasce come una applicazione dei metodi dell'Ingegneria del Software studiati a lezione, in particolare ci siamo concentrati sul realizzare un progetto utilizzando il  \textbf{modello a cascata.}


\chapter{Descrizione del Contesto}
\section{Funzionamento della Biblioteca}
\begin{itemize}
	\item Descrizion della Biblioteca di Recanati e del suo  Funzionamento
	\item Difficoltà riscontrate dalla biblioteca in questo contesto particolare
	
\end{itemize}
	
	
	
	
	
	
\textbf{\Huge \\Idee di Funzionalità}
\begin{itemize}
	\item Gestione dei Libri
	\item Registro degli ingressi e autoenticazione 
	\item Gestione della prenotazione dei posti (piantina?)
	\item Utente ritira il libro e da le nome e cognome e se vuole la mail per ricevere consigli sui libri
	\item NewsLetter in cui si consigliano i libri sulla base dell'analisi dei dati
	\item Secure Progrmming
	\item Reti Neurali	
\end{itemize}

Let's start out with the following theorem.



\section{Second section}

We begin our next section with the following central definition.


%---------------------------------------------------------------------------
\chapter{Second chapter}

\begin{summary}
  \blindtext
\end{summary}

\section{First section}
\Blindtext

\section{Second section}
\Blindtext

\section{Third section}
\Blindtext

%---------------------------------------------------------------------------
\chapter{Third chapter}

\begin{summary}
  \blindtext
\end{summary}

\section{First section}
\Blindtext

\section{Second section}
\Blindtext

\section{Third section}
\Blindtext

%---------------------------------------------------------------------------
% Bibliography
%---------------------------------------------------------------------------

\addcontentsline{toc}{chapter}{\textcolor{tssteelblue}{Literature}}
\printbibliography{}

%---------------------------------------------------------------------------
% Index
%---------------------------------------------------------------------------

\printindex

\end{document}

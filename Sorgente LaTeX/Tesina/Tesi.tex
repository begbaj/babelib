\documentclass{tstextbook}

\begin{document}

\tsbook{\Huge }
       {Chiara Gobbi \\ Daniele Benfatto \\ Davide De Zuane \\ Began Bajrami}
       {Antonio Muscerino}
       {2021}
       {xxxxx}{xxx--xx--xxxx--xx--x}{1.0}
       {Università Politecnica delle Marche}
       {Ancona}

%---------------------------------------------------------------------------
% Chapters
%---------------------------------------------------------------------------

%---------------------------------------------------------------------------

\begin{center}
	\textbf{\Large Abstract}
\end{center}
Questo progetto nasce come prova svolta per il corso di \textit{Ingegneria del Software} e rappresenta un primo approccio alla realizzazione professionale di software\\
Il progetto nasce come una applicazione dei metodi dell'Ingegneria del Software studiati a lezione, in particolare ci siamo concentrati sul realizzare un progetto utilizzando il  \textbf{modello a cascata.}


\chapter{Introduzione}
\section{Descrizione del Contesto}
L'ambiente che vogliamo informatizzare è una biblioteca.
La bibolioteca ha a disposizione diverse sale di studio, una sala per le conferenze, 


La biblioteca mette a disposizione un abbonamento che permette di diventare soci.
Una volta diventati soci è possibile usufruire di alcuni servizi aggiuntivi come: utilizzo di postazioni internet, consultazione dei giornali, partecipare a degli eventi esclusivi, servizio di newsletter.

Il bibliotecario dovrà occupparsi di differenti mansioni:
\begin{itemize}
	\item Creazione di Report;
	\item Gestione dei libri deteriorati;
	\item controllo qualità 
	\item Dovrà occupparsi della prenotazione dei libri;
	\item DOvrà occupparsi di registrare i soci della biblioteca;
\end{itemize}

\subsection{Domande}
\begin{itemize}
	\item descrizione generale del lavoro di bibliotecario (Mansioni svolte giornalmente)
	\item come è organizzata la sala studio 
	\item Come vengono gestise le prenotazioni
	\item organizzazione dei libri 
	
\end{itemize}


\subsection{Glossario}

\begin{tabular}{|l|l|l|r|}
	\hline
	Bibliotecario & o & sdsd\\
	\hline
	Newsletter & pubblicità personalizzate & gdfd \\
	Socio & a & aa & aaaa\\
	\hline
	a & aaa & aa & a\\
\hline
\end{tabular}

\begin{itemize}
	
	\item Descrizion della Biblioteca di Recanati e del suo  Funzionamento
	\item Difficoltà riscontrate dalla biblioteca in questo contesto particolare
	
\end{itemize}

\subsection{Requisiti}	
\begin{itemize}
	\item Inventario dei libri 
 	\item Anagrafica dei libri 
 	\item richiesta delle informazioni dell'utente
	nome del libro e controllo della disponibilità del libro
	consegna del libro
	\item Notifica della richiesta del libro se già preso 
	\item Stato del libro previa consegna e post consegna (controllo qualità)
	\item Lista con max posti che avvisa se si è raggiunto il limite \\
	\item Prenotazioni dei posti e gestione delle norme anti covid\\
	Piantina dei posti per la prenotazione (su richiesta del cliente sarà impossibile integrare in futuro una piantina dei posti o modifica delle funzionalità se vi è una variazione delle norme o della disposizione)\\
	\item Valutazione del libro da parte dei lettori (su base stelline)\\
	e comunicazione via email alla case editrice
	\item Tessera dei soci\\
	servizi aggiuntivi legati alla tessera dei soci (newsletter e altri)\\
	


\end{itemize}














Importazione di database esterno

Inventario dei libri 
 




richiest delle informazioni dell'utente
nome del libro e controllo della disponibilità del libro
consegna del libro

Notifica della richiesta del libro se già preso 

	
	
\textbf{\Huge \\Idee di Funzionalità}
\begin{itemize}
	\item Gestione dei Libri
	\item Registro degli ingressi e autoenticazione 
	\item Gestione della prenotazione dei posti (piantina?)
	\item Utente ritira il libro e da le nome e cognome e se vuole la mail per ricevere consigli sui libri
	\item NewsLetter in cui si consigliano i libri sulla base dell'analisi dei dati
	\item Secure Progrmming
	\item Reti Neurali	
\end{itemize}

Let's start out with the following theorem.



\section{Second section}

We begin our next section with the following central definition.


%---------------------------------------------------------------------------
\chapter{Second chapter}

\begin{summary}
  \blindtext
\end{summary}

\section{First section}
\Blindtext

\section{Second section}
\Blindtext

\section{Third section}
\Blindtext

%---------------------------------------------------------------------------
\chapter{Third chapter}

\begin{summary}
  \blindtext
\end{summary}

\section{First section}
\Blindtext

\section{Second section}
\Blindtext

\section{Third section}
\Blindtext

%---------------------------------------------------------------------------
% Bibliography
%---------------------------------------------------------------------------

\addcontentsline{toc}{chapter}{\textcolor{tssteelblue}{Literature}}
\printbibliography{}

%---------------------------------------------------------------------------
% Index
%---------------------------------------------------------------------------

\printindex

\end{document}

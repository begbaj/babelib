\documentclass{tstextbook}

\usepackage{array}
\usepackage{color}
\usepackage{colortbl}

\definecolor{LightCyan}{rgb}{0.77,0.86,0.94}
\definecolor{grigio}{rgb}{94,94,94}

\begin{document}

\tsbook{\Huge }
       {Chiara Gobbi \\ Daniele Benfatto \\ Davide De Zuane \\ Began Bajrami}
       {Antonio Muscerino}
       {2021}
       {xxxxx}{xxx--xx--xxxx--xx--x}{1.0}
       {Università Politecnica delle Marche}
       {Ancona}

%---------------------------------------------------------------------------
% Abstract
%---------------------------------------------------------------------------
\begin{center}
	\textbf{\Large Abstract}
\end{center}
Questo progetto nasce come prova svolta per il corso di \textit{Ingegneria del Software} e rappresenta un primo approccio alla realizzazione professionale di software.\\
In paricolare abbiamo reali

%---------------------------------------------------------------------------
% Chapters
%---------------------------------------------------------------------------
\chapter{Introduzione}

\begin{summary}
	In questo capitolo faremo un introduzione al progetto da realizzare e una panoramica del sistema da informatizzare, riportando le esperienze di intervista con il cliente.
\end{summary}

\section{Descrizione del Contesto}
L'ambiente che vogliamo informatizzare è la \textit{Biblioteca Comunale di Recanati}, che si trova in Corso Persiani 52 (MC).\\

Si tratta di una biblioteca di \textbf{circolazione,} inolre mette a disposizione degli utenti diversi servizi come: prestito di libri, postazioni internet, sale studio, sale conferenze e la consultazione di documenti di rilevante importanza(libri sulla storia del posto).\\

Le principali mansioni svolte abitualmente dai bibliotecari sono le attività di prestito e di riordinamento della scaffaltaura.\\
Il prestito dei libri prevede una specifica procedura: l'utente comunica il libro che vuole prendere in prestito, l'adetto ne  verifica la disponibilità e se lo è si dovrà occupare di emmettere un documento che certifichi il prestito.\\
Ogni biblioteca utilizzato una metodologia di \textbf{catalogazione} personale, quella di Recanati suddivide il catalogo dei libri seguendo una suddivisione in scaffali numerati. Ogni scaffale a sua volta viene suddiviso in palchetti(ognuno dei quali è indentificato da una lettera).
La posizione del singolo libro all'interno dello scaffale è identificato da un numero.\\
Ad ogni libro viene associato un inìdentificatore univoco che è il \textbf{numero d'inventario.}\\

La biblioteca di Recanati fa parte del \textbf{SBN} (Servizio Bibliotecario Nazionale). Tramite l'\textbf{OPAC SBN} gli utenti possono ricercare un documento e verificarne la disponibilità sul catalogo della Biblioteca.\\
I bibliotecari periodicamente devono occuparsi di aggiornare questo catalogo.\\

Il programma gestionale attualmente utilizzato dalla Biblioteca è il \textbf{Sebina Open Library}, il quale consente di svolgere ulteriori operazioni oltre a quelle già elencate.\\
Tuttavia essendo un Software generale risulta essere complicato per un impiegato medio e non adattabile alle esigenze personali di ogni singola biblioteca.\\
Per questo ci è stato richiesto di sviluppare un sistema che svolga le funzioni principali del sistema già in uso ma che sia più semplice ed intuitivo da utilizzare.\\

In questa situazione particolare la biblioteca si è ritrovata a dover gestire manualmente: la momentanea quarantena dei libri, la gestione degli spazi fisici in regime di distanziamento e dell'accesso agli spazi.\\
La \textit{momentanea quarantena} prevede che un libro restituito trascorra un intervallo di tempo (nel nostro caso 3 giorni) in una zona predisposta alla disinfezione e areata, dopodichè il libro torna disponibile.\\
L'\textit{accesso agli spazi},area studio e consultazione, è possibile solo previa prenotazione; gli adetti gestiscono queste prenotazioni manualmente e vorrebbero una funzionalità che ne faciliti la gestione.\\

Il Programma dovrà gestire anche \textbf{dati sensibili}, in quanto per poter usufruire dei servizi è necessario iscriversersi, gli utenti dovranno lirasciare il loro consenso al trattamento ai dati personali.\\

\newpage

\section{Glossario dei Termini}
Spieghiamo il significato di alcune parole specifiche del dominio.
\begin{center}
	\begin{tabular}{%
			>{\itshape}p{50mm}%
			>{\sffamily}p{20mm}%
			>{\sffamily}p{80mm}%
		}
		%Intestazione della Tabella
		\toprule
		\textbf{\large Termine} & \textbf{\large Tipo} & \textbf{\large Descrizione} \\
		%Corpo della Tabella
		\midrule
		\rowcolor{LightCyan}	
		\textbf{Bibliotecario} & Operatore & Il responsabile che si trova a dover utilizzare il software\\
		\textbf{Biblioteca di Circolazione} & & Effeutta prestito e consultazione di libri.  \\
		\rowcolor{LightCyan}
		\textbf{Consultazione} & &  Possibilità di leggere manoscritti all'interno della biblioteca\\
		\textbf{Newsletter} & Servizio & Un servizio di novirà che vengono comunicate agli utenti via e-mail.\\	
		\rowcolor{LightCyan}	
		\textbf{Numero d'inverntario} & & Codice univoco che identifica il libro\\
		\textbf{OPAC SBN} & Servizio & (On-line public access catalogue) E' il catalogo unificato, in rete, delle biblioteche\\
		\rowcolor{LightCyan}
		\textbf{Palchetti} & & Mensole di uno scaffale\\
		\textbf{SBN} & & Servizio Bibliotecario Nazionale, è la rete delle biblioteche italiane\\
		\rowcolor{LightCyan}
		\textbf{Socio} & Utente & Persona che si è iscritta alla biblioteca per usufruire dei servizi  \\
		
	\end{tabular}
\end{center}



%---------------------------------------------------------------------------
% Chapters
%---------------------------------------------------------------------------
\chapter{Analisi dei Requisiti}

\begin{summary}
  \blindtext
\end{summary}


%---------------------------------------------------------------------------
% Bibliography
%---------------------------------------------------------------------------

\addcontentsline{toc}{chapter}{\textcolor{tssteelblue}{Literature}}
\printbibliography{}

%---------------------------------------------------------------------------
% Index
%---------------------------------------------------------------------------

\printindex

\end{document}
